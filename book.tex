\documentclass{report}
\usepackage{amsfonts}
\usepackage{amsthm}
\usepackage{amsmath}
\newtheorem{theorem}{Theorem}
\DeclareMathOperator{\diff}{diff}
\title{\Huge\sc \textbf{PHILOSOPHI\AE}\\ NATURALIS\\ \textbf{PRINCIPIA}\\ MATHEMATICA}
\author{Al Caveman}
\begin{document}
\maketitle
\tableofcontents

\chapter{Preliminaries Because You Are an Idiot}
\textbf{NOTE: THIS CHAPTER IS PROBABLY WRONG I WILL FIX ONE DAY}

There seems to be a book written on fundamentals of geometry by some guy named Hiblert something, that might have a different view. Who cares.

\section{What is a Ruler}
A stick with things on it that tell you how long other things are. This is called \emph{The Axiom of Ruler}.

\section{What is a Dot?}
In a $d$-dimensional space, such as $\mathbb{R}^d$, a dot is a thing with these properties:
\begin{itemize}
\item Its position in the space $\mathbb{R}^d$ is defined by the vector $(x_i, x_2, \ldots, x_d)$ where for any $1 \le i \le d$, $x_i \in \mathbb{R}$.
\item If you look at the dot from the perspective of any specific dimension (e.g. $i=1,2, \ldots, d$), and put a ruler on the dot, you will see that its width $w_i$ is $0 < w_i < g$, where $g$ is any positive real number. Do you understand? It means the tiniest possible real number that is closest to 0 but not 0. Now go fuck yourself. In other words, a dot in $\mathbb{R}^d$ can be thought as a cube with $d$ faces. 
\end{itemize}

Some say that it is impossible to find $w_i$ such that for any positive g, $0 < w_i < g$. But I don't buy that. I think that there is, except that we cannot find it. There is a difference between something to not exist, and us not being able to find something. So I think the number exists, but we can't fine it. All we can do is define it as $w_i$.

Why should $w_i$ be positive instead of just 0? Because we use dots to define lines, and volumes. If $w_i$ for any dimension $i$ is zero, then all lines/volumes are also 0. Dead-end. We have to make it positive, but smallest positive to permit maximum granularity.


\section{What is a Line?}
A line is formed by putting a lot of dots next to each other, such that the gap between those lines is 0.

Suppose that we demand to create a line such that if we measure its length by a ruler (using axiom of ruler), its width is $x$. How many dots are there? Let $w$ be dots width, and their quantity be $y$ such that:
\[
    x = wy
\]

So there are $y$ many of them by definition. So $y = \frac{x}{w}$. We also know that by definition $w = 0$.

\chapter{Integration}
The goal here is to integrate the area under $f(x)$ from when $x = 0$ until we
reach $x = x_{end}$. What follows here is bunch of steps that shows my thinking
process because \emph{shadowdaemon} asked for it (and also because I am happy now).

\section{Areas of Lots of Extremely Tiny Rectangular Columns}
So to integrate $f(x) = x^2$ we have to keep summing extremely skinny
columns\footnote{\textbf{Note:} \emph{fefelix} of
\emph{freenode/\#gentoo-chat-exile} tried to look smart by attacking my rigor
by saying that the term \emph{skinny columns} is wrong and that it must be
replaced by the term \emph{infinitesimal} (facepalm moment here). He also tried
to look even smarter by using the phrase \emph{In the realm of $\mathbb{R}$}.
Obviously, any half-assed mathematician knows that $\mathbb{R}$ has only a
single infinitesimal which is zero. Yep, zero bitch. So the term
\emph{infinitesimal} is absolutely wrong in this context, thus even worse than
\emph{skinny columns}. The only exception is if \emph{fefelix} wishes to live
in the 1600s.} of, each of width $d$.
\[\begin{split}
  (x+d-x)(x)^2 + \\
  (x+2d-(x+d))(x+d)^2 + \\
  (x+3d-(x+2d))(x+2d)^2 + \\
  (x+4d-(x+3d))(x+3d)^2 + \\
  (x+5d-(x+4d))(x+4d)^2 + \\
  (x+6d-(x+5d))(x+5d)^2 + \\
  %
  (x+7d-(x+6d))(x+6d)^2 + \\
  (x+8d-(x+7d))(x+7d)^2 + \\
  (x+9d-(x+8d))(x+8d)^2 + \\
  (x+10d-(x+9d))(x+9d)^2 + \\
  (x+11d-(x+10d))(x+10d)^2 + \\
  \ldots
\end{split}\]



You see, if $d$ is extremely tiny (near zero), then we will have to sum an
infinite number of those tiny skinny areas. But for simplicity I put $\ldots$
instead.

We can simplify things:
\[\begin{split}
  d(x)^2 + \\
  d(x+d)^2 + \\
  d(x+2d)^2 + \\
  d(x+3d)^2 + \\
  d(x+4d)^2 + \\
  d(x+5d)^2 + \\
  %
  d(x+6d)^2 + \\
  d(x+7d)^2 + \\
  d(x+8d)^2 + \\
  d(x+9d)^2 + \\
  d(x+10d)^2 + \\
  \ldots
\end{split}\]

Let's expand those squares:
\[\begin{split}
  d(x)(x) + \\
  d(x+d)(x+d) + \\
  d(x+2d)(x+2d) + \\
  d(x+3d)(x+3d) + \\
  d(x+4d)(x+4d) + \\
  d(x+5d)(x+5d) + \\
  %
  d(x+6d)(x+6d) + \\
  d(x+7d)(x+7d) + \\
  d(x+8d)(x+8d) + \\
  d(x+9d)(x+9d) + \\
  d(x+10d)(x+10d) + \\
  \ldots
\end{split}\]

Let's multiply them square bitches:
\[\begin{split}
  dx^2 + \\
  d(x^2 + 2dx + d^2) + \\
  d(x^2 + 4dx + 4d^2) + \\
  d(x^2 + 6dx + 9d^2) + \\
  d(x^2 + 8dx + 16d^2) + \\
  d(x^2 + 10dx + 25x^2) + \\
  %
  d(x^2+12dx   +36d^2) + \\
  d(x^2+14dx   +49d^2) + \\
  d(x^2+16dx   +64d^2) + \\
  d(x^2+18dx   +81d^2) + \\
  d(x^2+20dx   +100d^2) + \\
  \ldots
\end{split}\]


Let's now multiply those bitches with $d$ so that the shit gets spread even
more:
\[\begin{split}
  dx^2 + \\
  dx^2 + 2d^2x + d^3 + \\
  dx^2 + 4d^2x + 4d^3 + \\
  dx^2 + 6d^2x + 9d^3 + \\
  dx^2 + 8d^2x + 16d^3 + \\
  dx^2 + 10d^2x + 25x^3 + \\
  %
  dx^2+12d^2x   +36d^3 + \\
  dx^2+14d^2x   +49d^3 + \\
  dx^2+16d^2x   +64d^3 + \\
  dx^2+18d^2x   +81d^3 + \\
  dx^2+20d^2x   +100d^3 + \\
  \ldots
\end{split}\]

\section{Approximating the Area}
Now this is a critical point. Below is basically saying that each row is an
approximation for the area under $f(x)$ from $x = 0$ till $x = x_{end}$. So the
$1^{st}$ one is a shit approximation where were approximate the area under that
curve by only one big fat column; so $d$ is so huge here, in fact $d =
x_{end}$.

Then, the 2nd line show a slightly less shit approximation where we approximate
the area under the curve by two fat ass rectangular columns. So here $d =
\frac{x_{end}}{2}$.

So the approximation of the area under the curve gets more and more accurate in
each line.
\[\begin{split}
  dx^2\\
  2dx^2 + 2d^2x + d^3\\
  3dx^2 + 6d^2x + 5d^3\\
  4dx^2 + 12d^2x + 14d^3\\
  5dx^2 + 20d^2x + 30d^3\\
  6dx^2 + 30d^2x + 55d^3\\
  7dx^2 + 42d^2x + 91d^3\\
  8dx^2 + 56d^2x + 140d^3\\
  9dx^2 + 72d^2x + 204d^3\\
  10dx^2 + 90d^2x + 285d^3\\
  11dx^2 + 110d^2x + 385d^3\\
  %dx^2 + 2d^2x + d^3\\
  %dx^2 + 4d^2x + 4d^3\\
  %dx^2 + 6d^2x + 9d^3\\
  %dx^2 + 8d^2x + 16d^3\\
  %dx^2 + 10d^2x + 25x^3\\
  %
  %dx^2+12d^2x   +36d^3 + \\
  %dx^2+14d^2x   +49d^3 + \\
  %dx^2+16d^2x   +64d^3 + \\
  %dx^2+18d^2x   +81d^3 + \\
  %dx^2+20d^2x   +100d^3 + \\
\end{split}\]

So basically, you see there is a pattern. The coefficient of the $1^{st}$ term
is easy peasy (just incrementing from 1 to $\infty$). The coefficient from the
$2^{nd}$ term is kinda interesting, it follows the equation $(i^2+i)$ where $i$
is the line number. Note that we start counting lines from 0. So the $1^{st}$
has $i = 0$ and the $2^{nd}$ line has $i = 1$, etc. Finally, the last term is
kinda cool, it follows the pattern $\frac{(i^2+i)(2i+1)}{6}$.

Now you may ask, how did I find these patterns? Well these are well known
number series. You can look them up in the On-Line Encyclopedia of Integer
Sequences\footnote{http://oeis.org/}.

So, the area under the curve of $f(x)$ from $x = 0$ up to $x_{end}$, by any
$d$ (and its corresponding $i$) is:
\[\begin{split}
  dx^2  +  (i^2+i)d^2x   +  \frac{(i^2+i)(2i+1)}{6}d^3\\
\end{split}\]

Now we are almost done. We know that $x = 0$, so we can cancel a few terms:
\begin{equation}\label{eq:dick}\begin{split}
  d0^2  +  (i^2+i)d^20   +  \frac{(i^2+i)(2i+1)}{6}d^3\\
  \frac{(i^2+i)(2i+1)}{6}d^3\\
\end{split}\end{equation}

Of course, we could've canceled those terms that multiply against zero earlier,
but I didn't for random reasons. I just didn't. That's the randomness of life.
But it's all mathematically correct as my caveman balls tell.

You can code a simple script that you give it $x_{end}$ and $d$, by which it
automatically finds $i = x_{end}/d$. You will notice that as $d$ gets smaller,
you end up approaching some limit after which reduction in $d$ does not cause
any change in the estimated area under the curve.

\section{The Precise Area Under The Bitch}
Now let's find the ultimate precision in the limit as $i \rightarrow
\infty$ which also means that $d \rightarrow 0$. But how about not?
Cause it's too hard to solve the limit when two variables are
approaching different limits.

To simplify the limits in an easier way, let's represent $i$ in terms of $d$
and $x_{end}$ as follows $i = x_{end}/d$. Then the same equation would become
as follows:
\[\begin{split}
  \frac{((x_{end}/d)^2+(x_{end}/d))(2(x_{end}/d)+1)}{6}d^3\\
  \frac{(\frac{x_{end}^2}{d^2}+\frac{x_{end}}{d})(\frac{2x_{end}}{d}+1)}{6}d^3\\
  \frac{(\frac{x_{end}^2}{d^2}+\frac{x_{end}}{d})(\frac{2x_{end}}{d}+1)}{6}d^3\\
  \frac{\frac{1}{d}(\frac{x_{end}^2}{d}+x_{end})(\frac{2x_{end}}{d}+1)}{6}d^3\\
  \frac{(\frac{x_{end}^2}{d}+x_{end})(\frac{2x_{end}}{d}+1)}{6}d^2\\
  \frac{\frac{2x_{end}^3}{d^2} +  \frac{x_{end}^2}{d} +  \frac{2x_{end}^2}{d} + x_{end}}{6}d^2\\
  \frac{\frac{2x_{end}^3d^2}{d^2} +  \frac{x_{end}^2d^2}{d} +  \frac{2x_{end}^2d^2}{d} + x_{end}d^2}{6}\\
  \frac{2x_{end}^3 +  x_{end}^2d +  2x_{end}^2d + x_{end}d^2}{6}\\
\end{split}\]

Now, it's super easy. We have to find the limit of that equation as a single
variable approaches 0 (we got rid of $i$). The equation becomes:
\[\begin{split}
  \frac{2x_{end}^3}{6}\\
  \frac{x_{end}^3}{3}\\
\end{split}\]

That's it. Integration re-invented bitch :) --- $\frac{x_{end}^2}3$.

Q.E. freaking DEE.


    \section{Generalizing That}
    Let's integrate $x^c$ where $x, c \in \mathbb{R}$. How easy is that? Let's
    try.

    So, back to the business of summing skinny columns:
    \[\begin{split}
        d \times d^c + \\
        d \times (2d)^c + \\
        d \times (3d)^c + \\
        d \times (4d)^c + \\
        d \times (5d)^c + \\
        \ldots
    \end{split}\]

    Simplified to:
    \[\begin{split}
        d \times d^c + \\
        d \times 2^c \times d^c + \\
        d \times 3^c \times d^c + \\
        d \times 4^c \times d^c + \\
        d \times 5^c \times d^c + \\
        \ldots
    \end{split}\]

    As we try to estimate that sum better we get:
    \[\begin{split}
        nd \times (\frac{n(n+1)}{2})^c \times d^c
    \end{split}\]




\chapter{Differentiation}
Here we want to find the slope of $f(x)$ at point $x$. This is easy so I won't
say much here.

Differentiate $x^2$.

$\frac{f(x+d) - f(x)}{x + d - x}$

$\frac{(x+d)^2 - x^2}{x + d - x}$

$\frac{(x+d)^2 - x^2}{d}$

$\frac{(x+d)(x+d) - x^2}{d}$

$\frac{x^2 + xd + xd + d^2 - x^2}{d}$

$\frac{x^2}{d} + \frac{2xd}{d} + \frac{d^2}{d} - \frac{x^2}{d}$

$\frac{x^2}{d} + 2x + d - \frac{x^2}{d}$

$2x + d$

Now, as $d \rightarrow 0$, it becomes $2x$. Done.

\section{Theorems}
My symbol for differentiation is $\diff$. That's it. Now, if $t, x, c$ are any
numbers, except $c$ is just a non-zero natural number, then:
\begin{theorem}
    $\diff tx^c = ctx^{c-1}$.
\end{theorem}


Okay let's do this... let's assume that $f(x) = tx^c$ for a wise reason that I
need not explain. Just trust me that I am doing the right thing.
\begin{proof}
    \[\begin{split}
        \diff f(x) &= \frac{f(x+d) - f(x)}{(x+d)-x}\\
                   &= \frac{f(x+d) - f(x)}{d}\\
                   &= \frac{t(x+d)^c - tx^c}{d}\\
                   &= \frac{t(\sum_{n=0}^c {c \choose c-n}x^{c-n}d^n) - tx^c}{d}\\
                   &= \frac{(\sum_{n=0}^c t {c \choose c-n}x^{c-n}d^n) - tx^c}{d}\\
                   &= \frac{t {c \choose c-0}x^{c-0}d^0 + (\sum_{n=1}^c t {c \choose c-n}x^{c-n}d^n) - tx^c}{d}\\
                   &= \frac{tx^{c} + (\sum_{n=1}^c t {c \choose c-n}x^{c-n}d^n) - tx^c}{d}\\
                   &= \frac{\sum_{n=1}^c t {c \choose c-n}x^{c-n}d^n}{d}\\
                   &= \sum_{n=1}^c \frac{t {c \choose c-n}x^{c-n}d^n}{d}\\
                   &= \sum_{n=1}^c t {c \choose c-n}x^{c-n}d^{n-1}\\
                   &= t {c \choose c-1}x^{c-1}d^{1-1} + \sum_{n=2}^c t {c \choose c-n}x^{c-n}d^{n-1}\\
                   &= tcx^{c-1} + \sum_{n=2}^c t {c \choose c-n}x^{c-n}d^{n-1}\\
    \end{split}\]

    Then:
    \[\lim_{d \rightarrow 0} tcx^{c-1} + \sum_{n=2}^c t {c \choose
    c-n}x^{c-n}d^{n-1} = tcx^{c-1}\]

    OMG it's the Q.E.D. baby!
\end{proof}


\chapter{Multivariate Stuff}
    \section{Functions that are sum of other linear functions}
        Let's say that we got linear functions $b(n)$ and $c(m)$. By
        definition, this means that:
        \begin{equation}
            b(n) = c_1 + w_1n
        \end{equation}
        \begin{equation}
            c(m) = c_2 + w_2m
        \end{equation}

        \begin{theorem}
            If $a(n, m) = b(n) + c(m)$, then $\lim_{(n,m) \rightarrow (x,y)} a(n,
            m) = \lim_{n \rightarrow x} b(n) + \lim_{m \rightarrow y} c(m)$
        \end{theorem}

        \begin{proof}
            Limits of $b$ and $c$:
            \begin{equation}
                \begin{split}
                    \lim_{n \rightarrow x} b(n) =& \lim_{n \rightarrow x} c_1 +
                    w_1n\\
                    =& c_1 + w_1x\\
                \end{split}
            \end{equation}
            \begin{equation}
                \begin{split}
                    \lim_{m \rightarrow y} c(m) =& \lim_{n \rightarrow y} c_2 +
                    w_2m\\
                    =& c_2 + w_2y\\
                \end{split}
            \end{equation}

            Limit of $a$ by theorem:
            \begin{equation}
                \begin{split}
                    \lim_{(n,m) \rightarrow (x,y)} a(n, m) &= c_1 + w_1x + c_2 + w_2y\\
                \end{split}
            \end{equation}
            
            By substitution we get expanded $a$:
            \begin{equation}
                a(n, m) = c_1 + w_1n + c_2 + w_2m
            \end{equation}

            Limit of expanded $a$:
            \begin{equation}
                \begin{split}
                    \lim_{(n,m) \rightarrow (x,y)} a(n, m) &= \lim_{(n,m)
                    \rightarrow (x,y)} c_1 + w_1n + c_2 + w_2m\\
                    &= c_1 + w_1x + c_2 + w_2y\\
                \end{split}
            \end{equation}

            Guess wat, limit of expanded $a$ is the same as the original one.
            Who could've thought?
        \end{proof}

    \section{Less Gay Stuff}
        Let's say that we wish to differentiate $f(x,y) = x^2/y$. What does it mean? The slope exactly at point $(x,y)$ that touches $f$? And by the way, my notation for that differentiation is: $\diff^f_{(x,y)} f(x,y)$ which means differentiate $f$ with respect to $(x,y)$.
        
        Obviously, there are infinitely many slopes cause we are dealing with surfaces (not curves). So in my logic, you 
\end{document}
